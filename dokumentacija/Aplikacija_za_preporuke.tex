\documentclass[]{foi} 
\usepackage[utf8]{inputenc}
\usepackage{lipsum}
\addbibresource{lib.bib}

\vrstaRada{\projekt}

\title{Aplikacija za preporuke}
\predmet{\predmetTBP}

\author{Hrvoje Lesar} 
\spolStudenta{\musko} 

\mentor{Bogdan Okreša Đurić}
\spolMentora{\musko} 
\titulaProfesora{dr. sc.}

\godina{2024}
\mjesec{Siječanj}

\indeks{0016133479}

\smjer{Organizacija poslovnih sustava}

\sazetak{Opsega od 100 do 300 riječi. Sažetak upućuje na temu rada, ukratko se iznosi čime se rad bavi, teorijsko-metodološka polazišta, glavne teze i smjer rada te zaključci.}
% abstract of 100 to 300 words.

% \kljucneRijeci{riječ; riječ; ...riječ; Obuhvaća $7\pm2$ ključna pojma koji su glavni predmet rasprave u radu.}
% keywords including 7 +/- 2 syntagms

\acrodef{VAS}{višeagentni sustav}


\begin{document}

\maketitle

\tableofcontents

\makeatletter \def\@dotsep{4.5} \makeatother
\pagestyle{plain}



\chapter{Uvod}

Završni ili diplomski rad studenta/studentice je konačni rezultat uloženog napora u završetak studija.
Obranom završnog ili diplomskog rada student/studentica stječe prava i obveze koje proizlaze iz završetka akademskog obrazovanja.
S ciljem osiguranja potpore studentima pri pisanju završnog/diplomskog rada, izrađen je ovaj predložak oblikovanja samog rada.

Načelna napomena o strukturi rada jest da se nazivi i struktura poglavlja obavezno definiraju u dogovoru s mentorom/mentoricom.
Sadržajna preporuka je da u uvodu treba opisati što je tema završnog/diplomskog rada,
zašto je tema značajna te koja je motivacija studenta/ studentice za odabir teme \cite{oraictolic2011AkademskoPismoStrategije}.



\chapter{Metode i tehnike rada}

U ovom poglavlju treba opisati koje će metode i tehnike biti korištene pri razradi teme,
kako su provedene istraživačke aktivnosti, koji su programski alati ili aplikacije korišteni.

\lipsum[1-2]


\chapter{Razrada teme}

Ovo je glavni dio rada u kojem treba razraditi temu, pojasniti istraživanja, prikazati rezultate i slično. Poželjno je na početku poglavlja dati kratki opis strukture poglavlja, kako bi čitatelj/čitateljica rada mogao/mogla lakše pratiti složenu cjelinu.



\section{Poglavlje druge razine }

\lipsum[6]



\subsection{Poglavlje treće razine}

\lipsum[2]



\subsection{Poglavlje četvrte razine}

\lipsum[4-5]



\chapter{Tehničke upute}

Tehničke upute u nastavku opisuju način tehničkog oblikovanja rada i navođenja literature.



\section{Upute za oblikovanje izgleda rada}

\subsection{Oblikovanje stranice}

\textbf{Stranice} su oblikovane korištenjem sljedećih parametara:

\begin{itemize}
	\item veličina i oblik papira je A4, okomito usmjerenje, margine 2,5 cm na svakoj strani;

	\item naslovna stranica rada se ne numerira;

	\item nakon naslovne stranice, sve sljedeće stranice do 1. Poglavlja se numeriraju rimskim brojevima, počevši od i;

	\item od 1. poglavlja nadalje, stranice se numeriraju arapskim brojevima;

	\item broj stranice treba pozicionirati desno 1,25 cm od dna stranice, font Arial 9.
\end{itemize}

\subsection{Tekst rada}

\textbf{Tekst} rada je oblikovan na sljedeći način:
\begin{itemize}
	\item u pisanju teksta koristite font Arial 11 pt, s proredom 1,5 te razmakom 0 pt prije i razmakom 6 pt poslije odlomka, pri čemu je prvi redak uvučen za 1,25 cm;

	\item u naslovima prve razine „3. Razrada teme“ koristite font Arial 18 pt, podebljano, prijelom stranice (svaki naslov prve razine treba biti na novoj stranici), s proredom 1,5 te razmakom 0 pt prije i razmakom 18 pt poslije odlomka;

	\item u naslovima druge razine „2.1. Naslov“ koristite font Arial 16 pt, podebljano, s proredom 1,5 te razmakom 18 pt prije i razmakom 12 pt poslije odlomka;

	\item u naslovima treće razine „2.1.1. Naslov“ koristite font Arial 14 pt, podebljano, s proredom 1,5 te razmakom 12 pt prije i razmakom 6 pt poslije odlomka;

	\item u naslovima četvrte razine „2.1.1.1. Naslov“ koristite font Arial 12 pt, podebljano, s proredom 1,5 te razmakom 6 pt prije i razmakom 6 pt poslije odlomka;

	\item ostalo značajno isticanje cjelina rada može biti istaknuto podebljanim i kurziv slovima, korištenjem fonta Arial 11 pt.
\end{itemize}

\subsection{Slike}

\textbf{Slike} u radu je potrebno oblikovati na sljedeći način:

\begin{itemize}
	\item opis slike navedite ispod slike uz numeraciju;

	\item za nazive slika koristite iste postavke fonta kao i za tekst, ali stavite opis slike u centrirani položaj;

	\item za oblikovanje same slike koristite font Arial 9 pt za tekst na slici;
	      ispred same slike umetnite jedan prazan redak (osim ako je slika pozicionirana na početku stranice);

	\item kod prijeloma stranice treba obratiti posebnu pozornost da opis slike, izvor i sama slika moraju biti na istoj stranici;

	\item slike je potrebno numerirati redom pojavljivanja u tekstu;

	\item ako je slika preuzeta iz drugog izvora, nakon navođenja opisa slike je potrebno dodati i referencu izvornog djela;

	\item dozvoljeno je napraviti vlastitu preradu slika, grafikona ili tablica na način da se zadrži isti smisao sadržaja, ali promijeni izgled, pri čemu je obavezno u opisu slike navesti referencu izvornog djela ovako: <opis slike> ; prema [X]

	\item dozvoljeno je preuzeti samo jednu sliku, grafikon ili tablicu u izvornom obliku iz istog izvora. Za doslovno preuzimanje većeg dijela sadržaja potrebno je ishoditi dozvolu nositelja autorskih prava;

	\item primjer označavanja slike prikazan je kod slike \ref{fig:podjela}.
\end{itemize}

\begin{figure}[]
	\centering
	\includegraphics[width=0.9\textwidth]{slike/slika.png}
	\caption{Osnovni koncepti modela umjetnog agenta; preuzeto iz \cite{russell2022ArtificialIntelligenceModern}}
	\label{fig:podjela}
\end{figure}

\subsection{Tablice}

\textbf{Tablice} rada je potrebno oblikovati sukladno ovim uputama:
\begin{itemize}
	\item opis tablice navedite iznad slike;

	\item za opise tablica koristite iste postavke fonta kao i za tekst, ali stavite opis tablice u centrirani položaj;

	\item za oblikovanje same tablice koristite font Arial 9 pt za tekst u tablici;

	\item tablice numerirajte redom pojavljivanja u tekstu;

	\item kod prijeloma stranice treba obratiti posebnu pozornost da opis tablice, izvor i sama tablica moraju biti na istoj stranici;

	\item ako je tablica preuzeta iz drugog izvora, nakon navođenja opisa tablice potrebno je navesti izvor, na isti način kako je opisano kod slika;

	\item primjer označavanja tablice možete vidjeti u nastavku (tablica \ref{tab:objekti}).
\end{itemize}

\begin{table}[h!]
	\centering
	\caption{Prikaz podataka o učestalosti pojavljivanja objekta; prema \cite{wooldridge2009IntroductionMultiAgentSystems}}
	\begin{tabularx}{0.66\textwidth}{|X|X|X|X|}
		\hline
		\cellcolor{gray!25} & \cellcolor{gray!25} & \cellcolor{gray!25} & \cellcolor{gray!25} \\
		\hline
		                    &                     &                     &                     \\
		\hline
		                    &                     &                     &                     \\
		\hline
	\end{tabularx}
	\\[10pt]
	\label{tab:objekti}
\end{table}

\subsection{Programski kod}

Za oblikovanje teksta koji je programski kod korišten je font Courier, veličine 10 pt, jednostruki prored, 6 pt iza odlomka, npr. HTML kod dijela zaglavlja početne web stranice FOI weba je prikazan kao isječak koda \ref{lst:dva}.

\begin{listing}
	\begin{minted}{xml}
<head>
  <meta http-equiv="Content-Type" content="text/html; charset=utf-8" />
  <link rel="shortcut icon" href="https://www.foi.unizg.hr/sites/default/files/favicon_0_1.ico" type="image/vnd.microsoft.icon" />
  <meta name="generator" content="Drupal 7 (http://drupal.org)" />
  <link rel="canonical" href="https://www.foi.unizg.hr/hr" />
  <link rel="shortlink" href="https://www.foi.unizg.hr/hr" />
  <!-- Set the viewport width to device width for mobile -->
  <meta name="viewport" content="width=device-width, initial-scale=1.0">
  <title>Dobro %*došli*) na FOI | FOI</title>...
</head>
    \end{minted}
	\caption{Primjer isječka koda}
	\label{lst:dva}
\end{listing}

U slučaju preuzetog programskog koda, za isti je nužno potrebno naznačiti izvor, kao u isječku koda \ref{lst:kod2}.

Ponekad palatali mogu stvarati probleme u opisima isječaka koda, pa ih je potrebno zamijeniti \LaTeX\ kodovima: \v{c}/\v{s}/\v{z}: \mintinline{text}{\v{c/s/z}}, \'{c}: \mintinline{text}{\'{c}}, \dj: \mintinline{text}{\dj}.


\begin{listing}
	\begin{minted}{python}
print("Ovo je preuzeti dio koda")
    \end{minted}
	\caption[Ovo je primjer koda koji je preuzet]{Ovo je primjer koda koji je preuzet iz \cite[str. 23]{russell2022ArtificialIntelligenceModern}}
	\label{lst:kod2}
\end{listing}



\subsection{Formule}

Za unos formula koristite editor za formule. Matematičke izraze moguće je pisati unutar teksta $E = mc^2$ ili izdvojeno:

$$
	a^2 + b^2 = c^2
$$

\subsection{Kratice}

Želite li koristiti kratice pojmova u tekstu, kad prvi put spominjete pojam potrebno je navesti puni naziv, a kraticu navesti u zagradi: informacijske i komunikacijske tehnologije (IKT). Nakon toga možete koristiti kratice u tekstu. Poželjno je u naslovima koristiti pune nazive.

Kratice je moguće definirati i korištenjem naredbe \mintinline{text}{\acrodef{}{}}, iz \mintinline{text}{acronym} paketa, prije početka dokumenta (naredba \mintinline{text}{\begin{document}}). Tako definirane kratice pozivaju se unutar dokumenta naredbom \mintinline{text}{\ac{}} ili \mintinline{text}{\Ac{}} za veliko početno slovo. Takva će naredba svoje prvo pojavljivanje u dokumentu zamijeniti potpuno ispisanim značenjem kratice i kraticom, dok će svako sljedeće pojavljivanje biti zamijenjeno samo postavljenom kraticom, npr.: \Ac{VAS} je sustav koji se sastoji od većeg broja agenata. Prilikom ponovnog korištenja naredbe, ispisana je samo kratica: \ac{VAS}.

\subsection{Strano nazivlje}

Strano nazivlje se u tekstu navodi u zagradi, napisano \textit{kurzivom}, nakon hrvatskog izraza, npr. analiza društvene mreže (engl. \textit{Social Network Analysis - SNA}).



\section{Navođenje literature}

Za navođenje literature u radu koristite \textbf{IEEE} stil. Važno je dosljedno primjenjivati odabrani stil u cijelom radu.

U popisu literature potrebno je navesti svu literaturu i samo literaturu koju ste koristili u tekstu.

Uz svaku preuzetu tvrdnju potrebno je navesti njezin izvor, tj. referencu. Reference se u tekstu navode tako da se uz citirani tekst navede izvor, sukladno načinu propisanom odabranim stilom i FOI preporukama za citiranje i referenciranje \cite{sveucilisteuzagrebufakultetorganizacijeiinformatike2017FOIPreporukeCitiranja}. Prilikom referenciranja knjiga, uvijek je potrebno navoditi i broj ili brojeve stranica.

Za citate i njihovo referenciranje koristite naredbu \mintinline{text}{\blockquote} kako slijedi. Citat će se automatski oblikovati kako je potrebno, ovisno o duljini citata. U kratkom citatu je pojašnjeno da bi se hibridna platforma za orkestraciju mogla u domeni videoigara koristiti
\blockquote[{\cite[str. 5]{schatten2020PlatformeZaOrkestraciju}}]{za implementaciju većeg broja inteligentnijih, a samim
	time i interesantnijih, suparnika.} Ono što slijedi nakon obavezno navedenog broja stranice uz izvor citata je primjer duljeg citata.

\blockquote[{\cite[str. 6]{schatten2020PlatformeZaOrkestraciju}}]{Iako je testiranje performansi MMO igara dobro implementirano na brojnim platformama, testiranje iskustva
	igranja (posebno u primjerice MMORPG igrama) velikog razmjera postaje iznimno složen zadatak, kao što
	je okvirno predstavljeno u (Schatten, Okreša Ðurić, Tomičić i Ivković, 2017). Ne samo da je potrebno testirati
	razne logičke zagonetke i zadatke koje igrač mora riješiti kako bi napredovao u igri, već i pojavljujuću interakciju među igračima koja bi mogla u potpunosti promijeniti rezultat igre. Korištenjem distribuirane platforme za orkestraciju bilo bi moguće dodavanje automatiziranih igrača po volji instanciranjem novih agenata. Štoviše, rezultati testiranja mogli bit biti dodatno
	analizirani dodavanjem agenata za analizu ili izvještavanje.}

Podaci o svim bibliografskim jedinicama nalaze se u \mintinline{text}{lib.bib} datoteci u \mintinline{text}{BibLaTeX} formatu. Bibliografske jedinice korištene u ovom dokumentu su:
\begin{itemize}
	\item knjige \cite{russell2022ArtificialIntelligenceModern,wooldridge2009IntroductionMultiAgentSystems,oraictolic2011AkademskoPismoStrategije},
	\item članak s konferencije \cite{okresaduric2019ModellingFormingTemporary},
	\item članci iz časopisa \cite{SchattenEtAl2016roadmap,rincon2017InfluencingPeopleSocial},
	\item web-stranica \cite{copeland2020ArtificialIntelligence}.
\end{itemize}



\chapter{Zaključak}

Ovdje treba sažeto rezimirati najvažnije rezultate razrade teme rada. Potrebno je sažeto opisati što je predmet rada \cite{copeland2020ArtificialIntelligence}, koje su metode, tehnike, programski alati ili aplikacije korištene u razradi rada te koje su pretpostavke dokazane, a koje opovrgnute. Sadržajno, ono što se u uvodu rada najavljuje i kasnije je obuhvaćeno u samom radu, moralo bi biti opisano u zaključnom dijelu kroz rezultate rada.

\lipsum[1-2]

\makebackmatter
% generira popis korištene literature, popis slika (ako je primjenjivo), popis tablica (ako je primjenjivo) i popis isječaka koda (ako je primjenjivo)

\appendices % ako nije potrebno, obrisati ili zakomentirati

\chapter{Prilog 1} % ako nije potrebno, obrisati ili zakomentirati

\chapter{Prilog 2} % ako nije potrebno, obrisati ili zakomentirati

\end{document}
