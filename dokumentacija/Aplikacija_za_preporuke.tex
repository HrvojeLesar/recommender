\documentclass[]{foi} 
\usepackage[utf8]{inputenc}
\usepackage{lipsum}
\addbibresource{lib.bib}

\vrstaRada{\projekt}

\title{Aplikacija za preporuke}
\predmet{\predmetTBP}

\author{Hrvoje Lesar} 
\spolStudenta{\musko} 

\mentor{Bogdan Okreša Đurić}
\spolMentora{\musko} 
\titulaProfesora{dr. sc.}

\godina{2024}
\mjesec{Siječanj}

\indeks{0016133479}

\smjer{Organizacija poslovnih sustava}

\sazetak{Opsega od 100 do 300 riječi. Sažetak upućuje na temu rada, ukratko se iznosi čime se rad bavi, teorijsko-metodološka polazišta, glavne teze i smjer rada te zaključci.}
% abstract of 100 to 300 words.

% \kljucneRijeci{riječ; riječ; ...riječ; Obuhvaća $7\pm2$ ključna pojma koji su glavni predmet rasprave u radu.}
% keywords including 7 +/- 2 syntagms

\acrodef{VAS}{višeagentni sustav}


\begin{document}

\maketitle

\tableofcontents

\makeatletter \def\@dotsep{4.5} \makeatother
\pagestyle{plain}



\chapter{Uvod}

Završni ili diplomski rad studenta/studentice je konačni rezultat uloženog napora u završetak studija.
Obranom završnog ili diplomskog rada student/studentica stječe prava i obveze koje proizlaze iz završetka akademskog obrazovanja.
S ciljem osiguranja potpore studentima pri pisanju završnog/diplomskog rada, izrađen je ovaj predložak oblikovanja samog rada.

Načelna napomena o strukturi rada jest da se nazivi i struktura poglavlja obavezno definiraju u dogovoru s mentorom/mentoricom.
Sadržajna preporuka je da u uvodu treba opisati što je tema završnog/diplomskog rada,
zašto je tema značajna te koja je motivacija studenta/ studentice za odabir teme \cite{oraictolic2011AkademskoPismoStrategije}.

\chapter{Polustrukturirane baze podataka}

\section{Polustrukturirani podaci}

Glavna ideja polustrukturiranih podataka je predstavljanje podataka kao vrstu strukture koja ima nalik na graf ili stablo.
Iako su dopušteni ciklusi između vrhova grafa, općenito takvu vrstu grafa možemo nazivati stablima \cite{buneman1997semistructured}.
Na slici \ref{fig:primjer_polustrukturiranih_podataka} je prikazan primjer modela podataka formaliziran u strukturu grafa.
Bridovi grafa su označeni tipom podatka ili više apstraktnim tipom koji se dalje grana. Prolaskom kroz graf moguće je
primijetiti da postoje dva različita načina prema kojima je opisan film. U prvom su glumci, filmska ekipa direktno nabrojeni,
dok u drugom postoji granjanje na glumce i kreditirane glumce. Zadnja grana grafa opisuje TV seriju koja opet ima drugačiju
strukturu u usporedbi s strukturom filmova.

\begin{figure}[h!]
	\centering
	\includegraphics[width=0.9\textwidth]{slike/graf_primjer.png}
	\caption{Primjer polustrukturiranih podataka u obliku grafa \cite{buneman1997semistructured}}
	\label{fig:primjer_polustrukturiranih_podataka}
\end{figure}

Polustrukturirani podaci su organizirani u sematičke entitete, ali nisu striktno u skladu s formalno strukturiranim strogim tipovima podataka.
Konceptualni model polustrukturiranih podataka mora sadržavati nekoliko svojstava kao štu su reprezentacija nepravilnih i heterogenih struktura
kao prikazanih na slici \ref{fig:primjer_polustrukturiranih_podataka}, hijerarhijske odnose uz nehijerarhijske vrste odnosa, kardinalnost,
relacije n-niza, poredak i reprezentaciju mješovitog sadržaja \cite{ganguly2012evaluations}.

\section{Vrste polustrukturiranih baza podataka}

Polustrukturirane baze podataka možemo podijeliti na četiri glavne vrste \cite{abramova2013nosql}:

\begin{enumerate}
	\item Key-Value store (Pohrana ključeva i vrijednosti); Podaci se pohranjuju kao skup ključeva i vrijednosti.
	      Ključevi su jedinstveni te se pristupa podacima povezivanjem ključa s vrijednosti.
	      Vrijednosti ne moraju striktno biti informacije, mogu biti drugi ključevi.
	\item Baze podataka temeljene na dokumentima; Mogu se definirati kao setovi ključeva i vrijednosti.
	      Svaki dokument je identificiran unikatnim ključem. Tip dokumenta je definiran prema znanim standardima
	      koji su večini slučajeva XML ili JSON. Pristup podacima moguć je korištenjem ključa ili određenih vrijednosti.
	\item Column-family (Obitelj stupaca); Podaci su postavljeni u stupce tj. sama struktura podataka i organizacija podatak
	      se sastoji od stupaca super-stupaca, i obitelji stupaca. Struktura baze je definirana kroz super-stupce i obitelj stupca.
	      Novi stupci se mogu dodavati po potrebi. Pristup podacima je moguć naznačujući obitelj stupca, ključ, stupac što će
	      rezultirati dohvačanjem vrijednosti.
	\item Graf baze podataka; Ova vrsta se koristi kad se podaci mogu prikazati kao graf, jedan primjer su društvene mreže
	      i veze između različitih korisnika.
\end{enumerate}

\section{MongoDB}

MongoDB je polustrukturirana baza temeljena na dokumentima. Dokumenti su grupirani u kolekcije prema njihovoj strukturi.
Dopušteno je spremanje dokumenta različitih struktura, no zbog boljih performansi preporuka je grupirati dokumente s
istom ili sličnom strukturom \cite{abramova2013nosql}. MongoDB koristi BSON kao format za spremanje dokumenata. BSON je kratica za Binary JSON.

Svaki novi dokument je moguće identificirati poljem \_id, te je za svaku kolekciju automatski kreiran indeks preko \_id polja.
Neke od najbitnijih karakteristika MongoDB-a su trajnost i moguće paralelno čitanje i pisanje podataka.
Trajnost je omogućeno kroz kreiranje replika baze, MongoDB koristi master-slave (gospodar-rob) mehanizam za replikaciju.
Omogućava definiranje jednog gospodara i jednog ili više robova. Gospodar je jedina replika kojoj je dozvoljeno pisanje i čitanje
dok robovi skuže samo kao sigurnosna kopija. U slučaju da se gospodar replika sruši, replika rob s najnovijim podacima je promovirana
u novog gospodara. Sve replike su asinkrone što znači da izvršena ažuriranja nisu odmah vidlija na svim replikama.
MongoDB postiže paralelno čitanje i pisanje podataka zaključavanjem podataka. Podaci koji se trenutno ažuriraju su zaključani
kako nebi bili pročitani zastarjeli podaci ili kako se nebi dogodilo više upisa u isto vrijeme te više nebi bilo moguće
odrediti valjanost upisanih podataka.

\section{Sustavi za preporuke}

Sustavi za preporuke se koriste za izradu kolekcije stavaka koje bi mogle interesirati određene korisnike.
Dizajn sustava ovisi o domeni problema, proizvoda, stavaka koje se žele preporučiti te dostupnosti podataka i posebnih
karakteristika prema kojima bi bilo moguće kreirati preporuku \cite{melville2010recommender}.
Sustavi za preporuke se razlikuju po načinu na koji analiziraju izvor podataka kako bi odredili
srodnost između korisnika i stavka koje se mogu koristiti za identifikaciju podudarnih parova.

Načini pristupa rješavanju problema preporuka moguće je kategorizirati u nekoliko glavnih kategorija \cite{lu2012recommender}:
\begin{enumerate}
	\item Kolaborativno filtriranje; Kolaborativno filtriranje radi na principu prikuplanja povratnih informacija od korisnika.
	      Povratne informacije su uglavnom u obliku ocjena za stavke u određenoj domeni. Iskorištavanjem sličnosti u ponašanju
	      ocjenjivača se određuje manji broj korisnika kojima kojima je moguće preporučiti neku stavku.
	\item Preporuke temeljene na sadržaju; Daju preporuke uspoređujući prikaze opisa sadržaja koji opisuju neku stavku s
	      prikazima sadržaja koji zanimaju korisnika.
	\item Hibridni pristupi; Pokušava kombinirati kolaborativno filtriranje i preporuke temeljene na sadržaju kako bi
	      preporuke bile što bolje i točnije.
\end{enumerate}

\chapter{Model baze podataka}

\section{Korišteni skup podataka}

Skup podataka koji je korišten kao temeljni je \texttt{goodreads-10k} skup. Skup sadrži oko deset tisuća knjiga i oko šest milijuna ocjena
knjiga, tagove i žanrove dodijeljene knjigama od strane korisnika. Inicijalni skup je raspoređen u nekoliko \texttt{.csv} datoteka i može
se prikazati sljedećim dijagramom:

\begin{figure}[h!]
	\centering
	\includegraphics[width=0.9\textwidth]{slike/goodbooks-10k_dijagram.png}
	\caption{Dijagram strukture podataka korištenog skupa podataka}
	\label{fig:struktura_podataka}
\end{figure}

Kroz dijagram na slici \ref{fig:struktura_podataka} je vidljivo da su podaci strukturirani na način pogodan za relacijske baze podataka.
Korišteni skup će biti direktno importiran u MongoDB no potrebno će ga biti transformirati u skup više pogodan za rad s polustrukturiranim bazama.
Kolekcije kreirane importiranjem podataka imaju nazive \texttt{tags}, \texttt{ratings}, \texttt{book\_tags}, \texttt{books}.

\section{Pretvorba podataka}

Sve navedene nardebe moguće je pokrenuti preko mongo shella u bazi podataka koja ima importiran prije definiran skup podataka.
Pretvorba podataka započinje smanjivanjem dostupnih broja tagova, tj. korištenjem samo onih najpopularnijih. Pošto su tagovi iz izvornog
skupa podataka definirani od strane korisnika mogu imati vrlo različite nazive i imati različita značenja za korisnike, time bi
najpopularniji tagovi morali filtrirati one manje korisne.

\begin{minted}[samepage]{javascript}
db.book_tags.agreggate([
    { $group: { _id: "$tag_id", totalCount: { $sum: "$count", }, }, },
    { $sort: { totalCount: -1, }, },
    { $limit: 300, },
    { $lookup: { from: "tags", localField: "_id", foreignField: "tag_id", as: "tag", }, },
    { $group: { _id: "$tag", }, },
    { $unwind: "$_id", },
    { $replaceRoot: { newRoot: "$_id", }, },
    { $out: "most_popular_tags", },
])
    \end{minted}
\captionof{listing}{Kreiranje najpopularnijih tagova}
\label{lst:tagovi}

Sljedeće se dodaju žanrovi knjigama (žanrovi su izvedeni iz tagova). Definirani upit prolazi kroz svaki tag,
provjerava koje knjige imaju dodijeljen trenutno selektirani tag, te ažurira sve dokumente u kolekciji \texttt{books}
sa tagom tj. žanrom koji pripada knjizi.

\begin{minted}[samepage]{javascript}
    db.most_popular_tags.find({}).map((tag) => {
        db.book_tags.agreggate([
            { $match: { tag_id: tag.tag_id } },
            { $project: { goodreads_book_id: 1 } }
        ]).map((tagIds) => {
            db.books.updateMany({goodreads_book_id: { $in: tagIds }}, { $addToSet: { genres: tag.name }})
        });
    })
\end{minted}
\captionof{listing}{Dodavanje žanrova knjigama}
\label{lst:zanrovi}

Zadnji korak agregiranje korisnika i njihovih ocjena knjiga u jednu kolekciju. U ovom koraku se svakom korisniku
dodijeljuje skup knjiga koje su ocjenili.

\begin{minted}[samepage]{javascript}
    db.ratings.agreggate([
        { $lookup: { from: "books", localField: "book_id", foreignField: "book_id", as: "book", }, },
        { $project: { _id: 1, user_id: "$user_id", rating: "$rating", book: { $arrayElemAt: ["$book", 0], }, }, },
        { $group: { _id: "$user_id", book_ratings: { $push: { rating: "$rating", book: "$book", }, }, }, },
        { $out: "users", },
    ])
\end{minted}
\captionof{listing}{Dodavanje žanrova knjigama}
\label{lst:korisnici}

Završne kolekcije u bazi se mogu reprezenrirati kao što je prikazano na sljedećoj slici \ref{fig:izvedene_kolekcije}.
Vidljivo je da su napravljene tri kolekcije \texttt{books}, \texttt{users} i \texttt{most\_popular\_tags}.
Kolekcija \texttt{books} sadržava knjige i ima na novo dodano polje \texttt{genres} što je lista žanrova knjige.
Kolekcija \texttt{users} su korisnici, identificirani samo svojim id-om, te svaki korisnik ima listu knjiga koje je
ocijenio, svaki element liste je ocjena i cijeli zapis ocjenjene knjige.

\begin{figure}[h!]
	\centering
	\includegraphics[width=0.9\textwidth]{slike/kolekcije.png}
	\caption{Izvedene kolekcije}
	\label{fig:izvedene_kolekcije}
\end{figure}


\section{Implementacija}

Sustav je implementiran u programskom jeziku \texttt{Go}. Za prikaz preporuka kreiran je web poslužitelj koji poslužuje web stranicu sa preporukama za
određenog korisnika. Konkretno se preporućuju knjige. Aplikacija za komunikacijom s MongoDB koristi Mongo Driver biblioteku.
Na web stranici postoji nekoliko vrsta preporuka:
\begin{itemize}
	\item preporuke temeljene na kolaborativnom filtriranju,
	\item preporuke određene prema najvišim ocjenama knjiga
	\item preporuke određene prema najvišim ocjenama knjiga prema određenom žanru
\end{itemize}
Korisnici imaju mogućnost dodavanja, izmjene, brisanja ocjena knjigama, time potencijalno mijenjanju preporuke za "slične" korisnike
i utječu na prosječnu ocjenu preporuka koje su određene prema najvišim ocjenama.

\subsection{Kolaborativno filtriranje}
\subsection{Najbolje ocjenjene knjige}
\subsection{Dodavanje, brisanje, promjena ocjena}

\chapter{Primjeri korištenja}

\chapter{Zaključak}

Ovdje treba sažeto rezimirati najvažnije rezultate razrade teme rada. Potrebno je sažeto opisati što je predmet rada \cite{copeland2020ArtificialIntelligence}, koje su metode, tehnike, programski alati ili aplikacije korištene u razradi rada te koje su pretpostavke dokazane, a koje opovrgnute. Sadržajno, ono što se u uvodu rada najavljuje i kasnije je obuhvaćeno u samom radu, moralo bi biti opisano u zaključnom dijelu kroz rezultate rada.

\lipsum[1-2]

\makebackmatter

\end{document}
